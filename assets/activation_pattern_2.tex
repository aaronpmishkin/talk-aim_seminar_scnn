%! TEX root = ../../main.tex

%% Illustration of cone decomposition. 

\begin{tikzpicture}[scale=1,
		declare function={
				cone_1(\x)= \x/2.25;
				cone_2(\x)= -\x;
				cone_3(\x)= -4*\x;
				bounds(\x)= \x - 10;
			}
	]
	\begin{axis}[width=\linewidth, height=8cm,
			axis lines=center, yticklabels={,,}, xticklabels={,,},
			ymin=-4, ymax=4, ytick={-5,...,5}, ylabel=$$, x axis line style={-},
				xmin=-6, xmax=6, xtick={-5,...,5}, xlabel=$$, y axis line style={-},
		]
		\addplot[name path=cone_1, domain=-6:6, samples=100, line width=1pt]{cone_1(x)};
		\addplot[name path=bounds, domain=-6:6, samples=100, line width=1pt]{bounds(x)};

		% add color fill to both cones.

		\addplot fill between[
				of = cone_1 and bounds,
				%split, % calculate segment
				every even segment/.style = {fill=green, fill opacity=0.3},
			];

		%% point labels
		% origin point
		\node[circle, fill, inner sep=1pt] at (axis cs:0,0) {};

		% active examples 
		\node[label=right:$x_1$, circle, fill, inner sep=0.5mm] at (axis cs:2,2) {};
		\node[label=right:$x_4$, circle, line width=0.25mm, draw=black, inner sep=0.5mm] at (axis cs:1.5,-2) {};

		% inactive examples 
		\node[label=right:$x_2$, circle, fill, inner sep=0.5mm] at (axis cs:-3,-2) {};
		\node[label=right:$x_3$, circle, line width=0.25mm, draw=black, inner sep=0.5mm] at (axis cs:-4,-1) {};
		\node[label=right:$x_5$, circle, line width=0.25mm, draw=black, inner sep=0.5mm] at (axis cs:0.5,3.5) {};


		% activation pattern
		\node[] at (axis cs:-3.25,2) {
			$D_j = \begin{bmatrix}
					\green{0} & 0       & 0       & 0         & 0       \\
					0         & \red{1} & 0       & 0         & 0       \\
					0         & 0       & \red{0} & 0         & 0       \\
					0         & 0       & 0       & \green{1} & 0       \\
					0         & 0       & 0       & 0         & \red{0} \\
				\end{bmatrix}
			$};

		% lines
		\draw [->, draw=red, line width = 0.4mm] (axis cs:0,0) -- (axis cs:1,-2.25) node[midway,above right] {$W_{1j}$};

	\end{axis}

\end{tikzpicture}%
